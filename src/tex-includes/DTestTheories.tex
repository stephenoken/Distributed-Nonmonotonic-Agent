%        1         2         3         4         5         6         7         8
%2345678901234567890123456789012345678901234567890123456789012345678901234567890

%% 
%% file:    DTestTheories.tex
%%
%% created: 26/11/99 by Andrew Rock
%%
%% purpose: A description of our scalable test theories 
%%

{

   \setlength{\arraycolsep}{0.25em}

This appendix specifies the scalable test theories
used to test the performance of \deimos\ system
components.

\subsection{Chain Theories} %%%%%%%%%%%%%%%%%%%%%%%%%%%%%%

\label{chainTheories}

Chain theories \chainTh\ start with
fact $a_{0}$ and continue 
with a chain of $n$ defeasible rules of the form
$a_{i-1} \Rightarrow a_{i}$. 
A proof of $+\partial a_{n}$ will use all of the
rules and the fact.

\[\mathbf{chain}(n) = \left\{
\begin{array}{rlll}
       &           & &          a_{0} \\
r_{1}: & a_{0}  & \Rightarrow & a_{1} \\
r_{2}: & a_{1}  & \Rightarrow & a_{2} \\
\multicolumn{4}{c}{\vdots}          \\
r_{n}: & a_{n-1} & \Rightarrow & a_{n}
\end{array}\right. \]

A variant \chainSTh\ uses only strict rules.

\[\mathbf{chain^{s}}(n) = \left\{
\begin{array}{rlll}
       &        & &             a_{0} \\
r_{1}: & a_{0} &  \rightarrow & a_{1} \\
r_{2}: & a_{1} &  \rightarrow & a_{2} \\
\multicolumn{4}{c}{\vdots}          \\
r_{n}: & a_{n-1} & \rightarrow & a_{n}
\end{array}\right. \]

\longForm{The implementation of functions that generate
chain theories is given in
section~\ref{chainTheoriesImpl}.}


\subsection{Circle Theories} %%%%%%%%%%%%%%%%%%%%%%%%%%%%%

\label{circleTheories}

Circle theories \circleTh\ consist of $n$
defeasible rules $a_{i} \Rightarrow a_{(i+1) \bmod n}$.

\[\mathbf{circle}(n) = \left\{
\begin{array}{rlll}
r_{0}: & a_{0} &  \Rightarrow & a_{1} \\
r_{1}: & a_{1} &  \Rightarrow & a_{2} \\
\multicolumn{4}{c} \vdots          \\
r_{n-1}: & a_{n-1} & \Rightarrow & a_{0}
\end{array}\right. \]

Any proof of $+\partial a_{i}$ will loop.
A variant \circleSTh\ uses only strict
rules.

\[\mathbf{circle^{s}}(n) = \left\{
\begin{array}{rlll}
r_{0}: & a_{0} &  \rightarrow & a_{1} \\
r_{1}: & a_{1} &  \rightarrow & a_{2} \\
\multicolumn{4}{c} \vdots          \\
r_{n-1}: & a_{n-1} & \rightarrow & a_{0}
\end{array}\right. \]

\longForm{The implementation of functions that generate
circle theories is given in
section~\ref{circleTheoriesImpl}.}


\subsection{Levels Theories} %%%%%%%%%%%%%%%%%%%%%%%%%%%%%

\label{levelsTheories}

Levels theories \levelsTh\ consist of a cascade
of $2n+2$ disputed conclusions $a_{i}$, $i \in [0..2n+1]$.
For each $i$, there are rules $\Rightarrow a_{i}$ and
$a_{i+1} \Rightarrow \neg a_{i}$. For each odd $i$ a priority
asserts that the latter rule is superior. A final rule
$ \Rightarrow a_{2n+2}$ gives uncontested support
for $a_{2n+2}$.

\[\mathbf{levels}(n) = \left\{\begin{array}{rlll}

   r_{0}:    & \{\}     & \Rightarrow & a_{0}          \\
   r_{1}:    & a_{1}    & \Rightarrow & \neg a_{0}     \\ \hline
   
   r_{2}:    & \{\}     & \Rightarrow & a_{1}          \\
   r_{3}:    & a_{2}    & \Rightarrow & \neg a_{1}     \\
   \multicolumn{4}{c}{r_{3} > r_{2}}                   \\ \hline
   
   r_{4}:    & \{\}     & \Rightarrow & a_{2}          \\
   r_{5}:    & a_{3}    & \Rightarrow & \neg a_{2}     \\ \hline

   \multicolumn{4}{c}{\vdots}                          \\ \hline

   r_{4n+2}: & \{\}     & \Rightarrow & a_{2n+1}       \\
   r_{4n+3}: & a_{2n+2} & \Rightarrow & \neg a_{2n+1}  \\
   \multicolumn{4}{c}{r_{4n+3} > r_{4n+2}}             \\ \hline
   
   r_{4n+4}: & \{\}     & \Rightarrow & a_{2n+2}

\end{array}\right.\]

A proof of $+\partial a_{0}$ will use every rule and 
priority. A variant \levelsMTh\ omits the priorities.

\longForm{The implementation of functions that generate
levels theories is given in
section~\ref{levelsTheoriesImpl}.}


\subsection{Teams Theories} %%%%%%%%%%%%%%%%%%%%%%%%%%%%%

\label{teamsTheories}

Teams theories \teamsTh\ consist of
conclusions $a_{i}$ which are supported by a team
two defeasible rules and attacked by another team
of two defeasible rules. Priorities ensure that each
attacking rule is beaten by one of the supporting rules.
The antecedents of these rules are in turn supported
and attacked by cascades of teams of rules.

\[\mathbf{teams}(n) = \mathbf{block}(a_{0},n)\]

\noindent where, if $p$ is a literal, and $r_{1}, \ldots
, r_{4}$ are new unique labels:

\[\mathbf{block}(p,0) =  \left\{ \begin{array}{rlll} 
      r_{1}: & \{\} &  \Rightarrow & p  \\
      r_{2}: & \{\} &  \Rightarrow & p  \\
      r_{3}: & \{\} &  \Rightarrow & \neg p  \\
      r_{4}: & \{\} &  \Rightarrow & \neg p  \\
      \multicolumn{4}{c}{r_{1} > r_{3}}       \\
      \multicolumn{4}{c}{r_{2} > r_{4}}       
   \end{array} \right.\]

\noindent and, if $n > 0$, 
$a_{1}, \ldots ,a_{4}$ are new unique literals,
and $r_{1}, \ldots ,r_{4}$ are new unique labels:

\[\mathbf{block}(p,n) =  
   \left\{ \begin{array}{rlll} 
      r_{1}: & a_{1} & \Rightarrow & p                \\
      r_{2}: & a_{2} & \Rightarrow & p                \\
      r_{3}: & a_{3} & \Rightarrow & \neg p            \\
      r_{4}: & a_{4} & \Rightarrow & \neg p            \\
      \multicolumn{4}{c}{r_{1} > r_{3}}             \\
      \multicolumn{4}{c}{r_{2} > r_{4}}             \\
      \multicolumn{4}{c}{\mathbf{block}(a_{1},n-1)} \\
      \multicolumn{4}{c}{\mathbf{block}(a_{2},n-1)} \\
      \multicolumn{4}{c}{\mathbf{block}(a_{3},n-1)} \\
      \multicolumn{4}{c}{\mathbf{block}(a_{4},n-1)} 
   \end{array} \right.\]

A proof of $+\partial a_{0}$ will use every rule and 
priority.

\longForm{The implementation of functions that generate
teams theories is given in
section~\ref{teamsTheoriesImpl}.}


\subsection{Tree Theories} %%%%%%%%%%%%%%%%%%%%%%%%%%%%%%

\label{treeTheories}

In tree theories \treeTh\, $a_{0}$ is at
the root of a $k$-branching tree of depth $n$ in which
every literal occurs once.

\[\mathbf{tree}(n,k) = \mathbf{block}(a_{0},n, k)\]

\noindent where, if $p$ is a literal, $n > 0$, $r$
is a new unique label, and $a_{1}, a_{2}, \ldots ,
a_{k}$ are new unique literals:

\[\mathbf{block}(p,n,k) =  \left\{ \begin{array}{rllllll} 
   r: &  a_{1}, & a_{2}, & \ldots , & a_{k} & \Rightarrow & p \\
   \multicolumn{7}{c}{\mathbf{block}(a_{1},n-1,k)}  \\
   \multicolumn{7}{c}{\mathbf{block}(a_{2},n-1,k)}  \\
   \multicolumn{7}{c}{\vdots}                        \\
   \multicolumn{7}{c}{\mathbf{block}(a_{k},n-1,k)}  
\end{array} \right.\]

\noindent and: 

\[\mathbf{block}(p,0,k) = \left\{ p \right.\]

A proof of $+\partial a_{0}$ will use every rule and 
fact.

\longForm{The implementation of functions that generate
tree theories is given in
section~\ref{treeTheoriesImpl}.}


\subsection{Directed Acyclic Graph Theories} %%%%%%%%%%%%%

\label{dagTheories}

In directed acyclic graph theories \dagTh,
$a_{0}$ is at the root of a $k$-branching tree of
depth $n$ in which every literal occurs $k$ times.

\[\mathbf{dag}(n,k) =  \left\{ \begin{array}{rllllll}
   & & & & & & a_{kn+1} \\
   & & & & & & a_{kn+2} \\
   \multicolumn{7}{c}{\vdots} \\
   & & & & & & a_{kn+k} \\
   r_{0}: & a_{1}, & a_{2}, & \ldots , & a_{k} & \Rightarrow & a_{0} \\
   r_{1}: & a_{2}, & a_{3}, & \ldots , & a_{k+1} & \Rightarrow & a_{1} \\
   \multicolumn{7}{c}{\vdots} \\
   r_{nk}: & a_{nk+1}, & a_{nk+2}, & \ldots , & a_{nk+k} & \Rightarrow & a_{nk} \\
\end{array} \right.\]

A proof of $+\partial a_{0}$ will use every rule and 
fact.

\longForm{The implementation of functions that generate
directed acyclic graph theories is given in
section~\ref{dagTheoriesImpl}.}


\subsection{Mix Theories} %%%%%%%%%%%%%%%%%%%%%%%%%%%%%%%%

\label{mixTheories}

In mix theories \mixTh\ there are $m$
defeasible rules for conclusion $p$ and $m$ defeaters
against $p$, where each rule has $n$ unique literals
as antecedents. Each antecedent literal can be strictly
established by a chain of strict rules of length $k$.
A proof of $+\partial p$ uses all the rules and facts.

\[\mathbf{mix}(m,n,k) =  \left\{ \begin{array}{rllllll}
   r_{1}: & a_{1,1}, & a_{1,2}, & \ldots, & a_{1,n} & \Rightarrow & p \\
   r_{2}: & a_{2,1}, & a_{2,2}, & \ldots, & a_{2,n} & \Rightarrow & p \\
   \multicolumn{7}{c}{\vdots} \\
   r_{m}: & a_{m,1}, & a_{m,2}, & \ldots, & a_{m,n} & \Rightarrow & p \\
   r_{m+1}: & a_{m+1,1}, & a_{m+1,2}, & \ldots, & a_{m+1,n} & \leadsto & \neg p \\
   r_{m+2}: & a_{m+2,1}, & a_{m+2,2}, & \ldots, & a_{m+2,n} & \leadsto & \neg p \\
   \multicolumn{7}{c}{\vdots} \\
   r_{2m}: & a_{2m,1}, & a_{2m,2}, & \ldots, & a_{2m,n} & \leadsto & \neg p \\
   \multicolumn{7}{c}{\mathbf{strictChain}(a_{1,1},k)} \\
   \multicolumn{7}{c}{\vdots} \\
   \multicolumn{7}{c}{\mathbf{strictChain}(a_{2m,n},k)}
\end{array} \right.\]

\noindent where:

\[\mathbf{strictChain}(a_{i,j},0) =  \left\{ \begin{array}{l}
   a_{i,j}
\end{array} \right.\]

\noindent or, if $k > 0$:

\[\mathbf{strictChain}(a_{i,j},k) =  \left\{ \begin{array}{rlll}
   & & & b_{i,j,1} \\
   r_{i,j,1}: & b_{i,j,1} & \Rightarrow & b_{i,j,2} \\
   r_{i,j,2}: & b_{i,j,2} & \Rightarrow & b_{i,j,3} \\
   \multicolumn{4}{c}{\vdots} \\
   r_{i,j,k-1}: & b_{i,j,k-1} & \Rightarrow & b_{i,j,k} \\
   r_{i,j,k}: & b_{i,j,k} & \Rightarrow & a_{i,j} 
\end{array} \right.\]

\longForm{The implementation of functions that generate
mix theories is given in
section~\ref{mixTheoriesImpl}.}


\subsection{Theory Sizes} %%%%%%%%%%%%%%%%%%%%%%%%%%%%%%%%

\label{theorySizeSection}

\begin{table*}[t]
   \centering

   \renewcommand{\arraystretch}[0]{1.5}

   \setMyFontSize

   \begin{tabular}{|c|c|c|c|c|c|}
      \hline
      \emph{theory} & \emph{facts} & \emph{rules}             & \emph{priorities}      & \emph{size}                            \\ 
      \hline
      \chainTh      & 1            & $n$                      & 0                      & $2n + 1$                               \\
      \chainSTh     & 1            & $n$                      & 0                      & $2n + 1$                               \\ 
      \hline
      \circleTh     & 0            & $n$                      & 0                      & $2n$                                   \\
      \circleSTh    & 0            & $n$                      & 0                      & $2n$                                   \\ 
      \hline
      \levelsTh     & 0            & $4n+5$                   & $n+1$                  & $7n+8$                                 \\
      \levelsMTh    & 0            & $4n+5$                   & 0                      & $6n+7$                                 \\ 
      \hline
      \teamsTh      & 0            & $4 \sum_{i=0}^{n} 4^i$   & $2 \sum_{i=0}^{n} 4^i$ & $10 \sum_{i=0}^{n-1} 4^i + 6 (4^{n})$  \\
      \hline 
      \treeTh       & $k^{n}$      & $\sum_{i=0}^{n-1} k^{i}$ & 0                      & $(k+1) \sum_{i=0}^{n-1} k^{i} + k^{n}$ \\
      \hline
      \dagTh        & $k$          & $nk + 1$                 & 0                      & $nk^{2} + (n+2)k + 1$                  \\
      \hline
      \mixTh        & $2mn$        & $2m + 2mnk$              & 0                      & $2m + 4mn + 4mnk$                      \\
      \hline
   \end{tabular}
      
   \caption{\setMyFontSize\label{theorySizeTable}
   Sizes of scalable test theories}

\end{table*}

A \deimos\ theory can be characterized by various metrics that
give an indication of the size or complexity of the theory.
These metrics might be used to estimate the memory required to
store a theory or estimate the time taken to respond to
queries to them.

Table~\ref{theorySizeTable} lists the formulae that predict these metrics for the 
scalable test theories described above. The metrics reported
are:

\begin{description}
   \item[facts] the number of facts in the theory;
   \item[rules] the number of rules in the theory;
   \item[priorities] the number of priorities in the theory; and
   \item[size] the overall ``size'' of the theory, defined as
      the sum of the numbers of facts, rules, priorities
      and literals in the bodies of all rules.
\end{description}

}
