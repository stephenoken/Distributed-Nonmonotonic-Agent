%        1         2         3         4         5         6         7         8
%2345678901234567890123456789012345678901234567890123456789012345678901234567890

%% 
%% script:  DefeasibleParser.tex
%%
%% created: 16/11/99 by Andrew Rock
%%
%% purpose: User's guide for DefeasibleParser.
%%

\module{DefeasibleParser} %%%%%%%%%%%%%%%%%%%%%%%%%%%%%%%%

\label{defParserUser}

The program {\tt DefeasibleParser} is a test program that
exercises the lexers and parsers required to parse 
a defeasible theory. It can be used as a quick syntax
checker for defeasible theory files.
This program can be run using the Hugs interpreter,
or compiled with GHC and run directly from the shell.


\submodule{Usage (GHC)} %%%%%%%%%%%%%%%%%%%%%%%%%%%%%%%%%%

Run the program with the command

\begin{alltt}
$ DefeasibleParser \emph{path1} \emph{path2} ... 
\end{alltt}

\noindent where \emph{path}, \emph{path2}, ... are the
paths to each of the theory files to be parsed.
For each file the program will display the
name of the file and either a syntax error message or, if
the file parsed correctly, the regenerated theory.
A check for cycles in the priority relation
is performed. If there are cycles, the priorities 
involved are printed.
If there are no cycles an attempt is made to remove
all variables by generating ground instances of them
using all of the constants appearing in the theory. The
grounded theory is printed.

If no paths are
supplied on the command line, then standard input will
be read and parsed.


\submodule{Usage (Hugs)} %%%%%%%%%%%%%%%%%%%%%%%%%%%%%%%%%

Load the script {\tt DefeasibleParser.lhs} into the Hugs
interpreter. To test the parser on one description file,
type the expression

\begin{alltt}
Hugs> run1 "\emph{path}"
\end{alltt}

\noindent where \emph{path} is the path to the
theory file. To test the parser on a list of files,
type the expression

\begin{alltt}
Hugs> run ["\emph{path1}", "\emph{path2}", ... ]
\end{alltt}

\noindent Standard input will not be parsed if that
list is empty, otherwise the program will then behave
as described for GHC.
