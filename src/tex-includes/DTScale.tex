%        1         2         3         4         5         6         7         8
%2345678901234567890123456789012345678901234567890123456789012345678901234567890

%% 
%% script:  DTScale.tex
%%
%% created: 29/11/99 by Andrew Rock
%%
%% purpose: User's guide for DTScale.
%%

\module{DTScale} %%%%%%%%%%%%%%%%%%%%%%%%%%%%%%%%%%%%%%%%%

\label{DTScaleUser}

The program {\tt DTScale} is used for the generation
of scalable test theories and for measuring the time
required for proofs using them.

This program can be run using the Hugs interpreter,
or compiled with GHC and run directly from the shell.
Execution time measurement is only possible using
the GHC compiled version of this program.

\submodule{Usage (GHC)} %%%%%%%%%%%%%%%%%%%%%%%%%%%%%%%%%

Compile the program by typing {\tt make DTScale}.
Run the program by typing a command of the form:

\begin{alltt}
$ DTScale  {\it options theory-name size...} 
\end{alltt}

\noindent where the options are:

\begin{description}
   \item[{\tt -t}] Print the theory in \deimos\ syntax and
      terminate without attempting a proof.
   \item[{\tt -tp}] Print the theory in \dprolog\ syntax and
      terminate without attempting a proof.
   \item[{\tt -td}] Print the theory in \delores\ syntax and
      terminate without attempting a proof.
   \item[{\tt -m}] Print the computed metrics (defined
      in section \ref{theorySizeSection}) for the
      theory before proving it.
   \item[{\tt -e} {\it prover}] Use the named {\it prover}
      engine. See tables \ref{dProverTable} and 
      \ref{flatHistProversTable} for the names of the
      provers that are available. The default prover is
      \verb"nHl".
   \item[{\tt -o}] Don't use the faster array-based theory
      representation.
\end{description}

\noindent Example:

\begin{alltt}
$ DTScale -t mix 100 10 5
\end{alltt}

When a proof is requested, statistics about the size of the
theory, the number of goals and the time required
for proof are printed.

The theory and the tagged
literal to use are specified by {\it theory-name} 
and {\it size}. The mapping from name to theory 
is given in table~\ref{theoryNameTable}. The scalable
test theories are described in detail in 
appendix~\ref{scalableAppendix}. 

\begin{table}[h]

\setMyFontSize

\centering

\renewcommand{\arraystretch}[0]{1.5}

\begin{tabular}{|l|l|l|} \hline

\emph{theory} & \emph{theory name} & \emph{smallest size} \\ \hline
 \chainTh   & {\tt chain}    & {\tt 0} \\ \hline
 \chainSTh  & {\tt chains}   & {\tt 0} \\ \hline
 \circleTh  & {\tt circle}   & {\tt 1} \\ \hline
 \circleSTh & {\tt circles}  & {\tt 1} \\ \hline
 \levelsTh  & {\tt levels}   & {\tt 0} \\ \hline
 \levelsMTh & {\tt levels-}  & {\tt 0} \\ \hline
 \teamsTh   & {\tt teams}    & {\tt 0} \\ \hline
 \treeTh    & {\tt tree}     & {\tt 1 1} \\ \hline
 \dagTh     & {\tt dag}      & {\tt 1 1} \\ \hline
 \mixTh     & {\tt mix}      & {\tt 1 0 0} \\ \hline
\end{tabular}

\caption{\setMyFontSize\label{theoryNameTable} Names
for specifying scalable test theories, and the smallest
size parameters permitted for each theory.}

\end{table}

\submodule{Usage (Hugs)} %%%%%%%%%%%%%%%%%%%%%%%%%%%%%%%%%%%%%

Load the script {\tt DTScale.lhs} into the Hugs
interpreter. At the Hugs prompt, type the expression 
{\tt run} {\it args}, where  {\it args} is a string
containing the command line arguments as described 
above for the compiled version. Example:

\begin{alltt}
Hugs> run "-p nhlt tree 5 3"
\end{alltt}

