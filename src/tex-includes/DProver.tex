%        1         2         3         4         5         6         7         8
%2345678901234567890123456789012345678901234567890123456789012345678901234567890

%% 
%% script:  DProver.tex
%%
%% created: 16/11/99 by Andrew Rock
%%
%% purpose: User's guide for DefeasibleParser.
%%

\module{DProver} %%%%%%%%%%%%%%%%%%%%%%%%%%%%%%%%%%%%%%%%%
\label{DProverUser}

The program {\tt DProver} is the query answering
prover with the simplest (and slowest) implementation. 
This program is maintained as a test-bed for new 
features as it is simpler and quicker to 
modify than the other prover programs constituting
\deimos. Current features available to this prover,
but not to others, include:

\begin{itemize}
   \item provers with well-founded semantics; and
   \item run-files.
\end{itemize}

This program can be run using the Hugs interpreter,
or compiled with GHC and run directly from the shell.



\submodule{Usage (GHC)} %%%%%%%%%%%%%%%%%%%%%%%%%%%%%%%%%%

Run the program by typing a command of the form:

\begin{alltt}
$ DProver\it options\tt [\it{theory-file-name}\tt [\it{tagged-literal}\tt]]
\end{alltt}

\noindent where the options are:

\begin{description}
   \item[{\tt -t}] Print the theory in \deimos\ syntax and
      terminate.
   \item[{\tt -tp}] Print the theory in \dprolog\ syntax
      and terminate.
   \item[{\tt -td}] Print the theory in \delores\ syntax
      and terminate.
   \item[{\tt -e} {\it prover}] Use the named {\it prover} engine.
      See table~\ref{dProverTable} for the names of the
      prover engines that are available. The default prover engine is
      \verb"nhlt".
   \item[{\tt -r} {\it run-file}] Use the named {\it run-file} to
      generate a truth table and terminate.
\end{description}

\noindent If a theory file name is supplied on the
command line, that theory will be loaded. Otherwise when the
program starts it will prompt for the name of a theory file
to load. If there is a tagged literal supplied on the command
line, then that proof will be attempted and the program
will terminate upon its completion. If the {\tt -r} option
is specified and a {\it run-file} name is supplied, then
all the proofs specified by the runfile are attempted,
and then a truth table will be printed. Otherwise the
program will prompt for and handle commands.

When a theory is loaded it is parsed and checked for
consistency. If these checks fail an error message will
be printed and another file name promped for.

When a theory has been loaded successfully, the program
prompts for commands with \verb"|-". The following commands
are accepted:

\begin{description}
   \item[{\tt ?}] Print the list of commands.
   \item[{\tt q}] Quit the program.
   \item[{\tt t}] Print the theory in \deimos\ syntax.
   \item[{\tt tp}] Print the theory in \dprolog\ syntax.
   \item[{\tt td}] Print the theory in \delores\ syntax.
   \item[{\tt f}] Forget the history of subgoals
      accumulated so far.
   \item[{\tt e}] Identify the current prover engine.
   \item[{\tt e~}{\it engine}] Select a prover {\it engine}.
   \item[{\tt l~[{\it file-name}]}] Load a new theory file
     [named {\it file-name}].
   \item[{\it tagged-literal}] Answer {\it tagged-literal}
      by attempting a proof.
   \item[{\tt r~[{\it run-file}]}] Run the named run-file, 
      printing a table of results.
\end{description}

Tagged literals are described in 
section~\ref{taggedLitSection}. The prover engines that can be 
selected with the {\tt e} command are listed in 
table~\ref{dProverTable}. The different provers feature
combinations of goal counting, avoiding recomputation
by maintaining a history of prior results, loop detection,
well-founded semantics, 
and trace printing. The default prover is \verb"nhlt".

\begin{table}[h]

\setMyFontSize

\centering

% \renewcommand{\arraystretch}[0]{1.5}

\begin{tabular}{|c|c|c|c|c|c|} \hline

\emph{prover} & \emph{counts} & \emph{keeps}   & \emph{detects} & \emph{well-}   & \emph{prints} \\ 
\emph{name}   & \emph{goals}  & \emph{history} & \emph{loops}   & \emph{founded} & \emph{trace}  \\ \hline
\verb"-"      &               &                &                &                &               \\ 
\verb"n"      & $\bullet$     &                &                &                &               \\ 
\verb"nh"     & $\bullet$     & $\bullet$      &                &                &               \\ 
\verb"nhl"    & $\bullet$     & $\bullet$      & $\bullet$      &                &               \\
\verb"nhlw"   & $\bullet$     & $\bullet$      & $\bullet$      & $\bullet$      &               \\
\verb"t"      &               &                &                &                & $\bullet$     \\ 
\verb"nt"     & $\bullet$     &                &                &                & $\bullet$     \\ 
\verb"nht"    & $\bullet$     & $\bullet$      &                &                & $\bullet$     \\ 
\verb"nhlt"   & $\bullet$     & $\bullet$      & $\bullet$      &                & $\bullet$     \\ 
\verb"nhlwt"  & $\bullet$     & $\bullet$      & $\bullet$      & $\bullet$      & $\bullet$     \\ \hline
\end{tabular}

\caption{\setMyFontSize\label{dProverTable}DProver 
provers.}

\end{table}


\submodule{Usage (Hugs)} %%%%%%%%%%%%%%%%%%%%%%%%%%%%%%%%%

Load the script {\tt DProver.lhs} into the Hugs
interpreter. At the Hugs prompt, type the expression 

\begin{alltt}
Hugs> run "\emph{options} [\emph{theory-file-name} [\emph{tagged-literal}]]"
\end{alltt}

\noindent The program then behaves as descibed for GHC.


\submodule{Run-files} %%%%%%%%%%%%%%%%%%%%%%%%%%%%%%%%%

A theory may be tested by augmentation by combinations of extra facts,
generating a summary table of results. {\tt DProver}
reads a file, a {\it run-file} to specify the combinations
of facts to test with and the proofs to attempt.

A run-file consists of a sequence of statements
that specify the literals to assert as facts,
the combinations of literals to ignore, and the 
proofs to attempt for each combination of inputs.

The syntax of a run-file is summarized as follows.

\begin{Verbatim}
run-file ::= {(input | ignore | output) "." }

input ::= "input" "{" literal {"," literal} "}"

ignore ::= "ignore"  "{" literal {"," literal} "}"

output ::= "output" "{" taggedLiteral "}"
\end{Verbatim}

All literals in a run-file must be grounded.
Comments are permitted, with the same syntax
as for theory files.

An {\tt input} statement usually contains one
literals. If two or more literals are present
in a single input statement,
then they are mutually exclusive. Examples are
shown in table~\ref{inputTable}.
An {\tt ignore} statement rules out specific
combinations of facts. An example is shown in 
table~\ref{inputTable}.
An {\tt output} statement specifies a proof to attempt
for each combination of literals.
A run-file will produce a summary table of results.
The results will be abbreviated as shown in table~\ref{abbrTable}.

\begin{table}[h]

\setMyFontSize

{

\centering

\begin{tabular}{|l|l|}                                                              \hline
\emph{statements}                                     & \emph{facts generated}   \\ \hline
{\tt input\char`\{a\char`\}. input\char`\{b\char`\}.} & {\tt ~a. ~b.}            \\
                                                      & {\tt ~a. \~{}b.}         \\
                                                      & {\tt \~{}a. ~b.}         \\
                                                      & {\tt \~{}a. \~{}b.}      \\ \hline
{\tt input\char`\{a, b\char`\}.}                       & {\tt ~a. \~{}b.}         \\
                                                      & {\tt \~{}a. ~b.}         \\ \hline
{\tt input\char`\{a, \~{}b\char`\}.}                   & {\tt ~a. ~b.}            \\
                                                      & {\tt \~{}a. \~{}b.}      \\ \hline
{\tt input\char`\{a\char`\}. input\char`\{b\char`\}. ignore\char`\{a, \~{}b\char`\}.} 
                                                      & {\tt ~a. ~b.}            \\
                                                      & {\tt \~{}a. ~b.}         \\
                                                      & {\tt \~{}a. \~{}b.}      \\ \hline
\end{tabular}

}

\caption{\label{inputTable} Example input and ignore statements
and the combinations of facts generated.}

\end{table}

\begin{table}[h]

\setMyFontSize

{

\centering

\begin{tabular}{|l|l|}                 \hline
\emph{Result} & \emph{abbreviation} \\ \hline
Proved        & {\tt P}             \\
Not Proved    & {\tt N}             \\
Loops         & {\tt L}             \\ \hline
\end{tabular}

}

\caption{\label{abbrTable} Abbreviated proof results.}

\end{table}
